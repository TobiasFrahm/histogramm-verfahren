% !TeX encoding = utf8
%\documentclass[conference]{IEEEtran}
\documentclass[conference,10pt,a4paper]{IEEEtran}  
\IEEEoverridecommandlockouts
% The preceding line is only needed to identify funding in the first footnote. If that is unneeded, please comment it out.
%\pdfminorversion=7
\usepackage{stfloats} 
\usepackage{amsmath,amssymb,amsfonts}
\usepackage{algorithmic}
\usepackage{graphicx}
\usepackage{textcomp}
\usepackage{xcolor} 
\usepackage{cancel}


%\usepackage{helvet}
%\usepackage{tikz}
%\usepackage[scaled]{uarial}

\renewcommand*\familydefault{\sfdefault}

\usepackage[T1]{fontenc}
\usepackage[utf8]{inputenc}




\definecolor{color1}{RGB}{200,200,200}

\usepackage[hidelinks 	= 	true, 
draft 		=	false,	% all hypertext options are turned off
final 		=	true, 	% all hypertext options are turned on
raiselinks 	= 	true,	% Allow links to reflect real height - e.g. with pictures
breaklinks	=	false,	% false=Do not break a line within a link
backref		=	false,	% false=no backlinks in the bibliography
pagebackref =	false,	% false=no pagebacklinks in bibliography
linktocpage =	true,	% true=makes page-no linked and not text in TOC,LOF and LOT
%			colorlinks	=	true,	% true=Colors the text of links and anchors
colorlinks	=	false,	% true=Colors the text of links and anchors
%			linkcolor 	=	red, 	% Color for normal internal links.
%			anchorcolor =	black, 	% Color for anchor text.
%			citecolor 	=	green, 	% Color for bibliographical citations in text.
%			filecolor 	=	cyan, 	% Color for URLs which open local files.
%			menucolor 	=	red,	% Color for Acrobat menu items.
%			runcolor 	= filecolor,% Color for run links (launch annotations).
%			allcolors 	= 	red,	% Set all color options (without border and field options).
allcolors   =   color1, 
urlcolor 	=	color1,	% Color for linked URLs.
allbordercolors  =color1,	% Set all colors
%			citebordercolor =green,	% cite color
%			filebordercolor =cyan,	% file color
%			linkbordercolor =red,	% link color
%			urlbordercolor  =blue,	% url color
%			pdfborder 			= 0 0 0,% weder farbige links noch umrandung			
%			menubordercolor 	= red,	% set menu color
%			frenchlinks 		= false,% Use small caps instead of color for links.
bookmarks			= true,	% true=Bookmarks are added to the pdf
bookmarksopen 		= false,% false=Bookmarks not open when opening the pdf
bookmarksnumbered 	= true,	% true= Eintraege sind nummeriert 
%			pdfpagemode	=	FullScreen,	% File is opened in Full Screen
%			pdfstartview=	fit,	% Fit size when opening pdf
pdfpagelabels=	true,	% true = Roemische Zahlen usw werden dargestellt,
% false= fortlaufende nummerierung
]{hyperref}
 
 

 
 
% \title{Charakterisierung einer Schraubenverbindung mittels magnetischem Sensor-Array}
%\title{Characterization of a Bolted Joint with a Magnetic Sensor Array}
%\title{Magnetic Sensor Array for Determination of the Applied Preload of a Bolted Joint}
\title{\fontsize{16pt}{18pt}\selectfont\bfseries Histogramm-Verfahren für die Signalaussteuerung bei der Impedanzspektroskopie für Fahrzeugbatterien\\[12pt]
	\fontsize{10pt}{11.5pt}\selectfont\normalfont\itshape
	$\underline{\text{Tobias Frahm}}$ \textsuperscript{1}, 
		Florian Rittweger \textsuperscript{1}, Thorben Schüthe \textsuperscript{1}, Karl-Ragmar Riemschneider \textsuperscript{1} \\~\\
 \textsuperscript{1}Hochschule für Angewandte Wissenschaften Hamburg, Berliner Tor 7, 20099 Hamburg, GER \\
	
}
 

 

\usepackage{booktabs}
\usepackage{multirow} 
\usepackage{amsmath}
\usepackage{siunitx} 
\sisetup{locale = DE,  
	separate-uncertainty,  
	%	range-units = brackets,  
	range-units= repeat, 
	range-phrase= ~bis~ ,
	list-units = single,  
	%	per-mode=symbol,
	per-mode=symbol-or-fraction,
	round-precision=3}

\DeclareSIUnit\Nm{Nm}
\DeclareSIUnit\Nm{Nm}
\usepackage[left=2.5cm,right=2.5cm,top=2.5cm,bottom=2.5cm]{geometry}

\usepackage{mathtools}
\usepackage{xspace}
%\usepackage{tools-overview}


\usepackage[main=ngerman]{babel} 
\usepackage[babel]{csquotes} 
%\usepackage[]{caption} 
%\captionsetup{format=} 
\usepackage[figurename=Abb.,
			tablename=Tab.,
			format=hang,
			textfont=it,
			labelfont=it,
			labelsep=quad,
			skip=1em]{caption}  
%\setlength{\belowcaptionskip}{-1em}



\usepackage[%
style		= ieee,    
defernumbers= true,      % to set different numeric types -e.g. A1...A2 B1...B2
sortcase	= false,%		% false- keine unterscheidung zwischen gross und kleinschrift 
bibencoding	= utf8,%
doi 		= true, 
isbn 		= false,
subentry	= false,
url 		= false,  
backend		= biber,
language    = {ngerman},
maxcitenames=2,
mincitenames=2,
]{biblatex}				% um 8-bit zeichen zu erkennen, fuer umlaute 

\DeclareFieldFormat{url}{Available\addcolon\space\url{#1}}
\DefineBibliographyStrings{german}{
	andothers = {{et\,al\adddot}},
	and = \&,
	page = {\:}
} 
\addto\extrasenglish{\languageshorthands{ngerman}} 
\usepackage{upgreek}
\usepackage{ltablex}




%===========================================
% Momente 
%===========================================
\newcommand{\MK}{\ensuremath{M_\mathrm{K}}\xspace}
\newcommand{\MA}{\ensuremath{M_\mathrm{A}}\xspace}
\newcommand{\MG}{\ensuremath{M_\mathrm{G}}\xspace}
\newcommand{\MT}{\ensuremath{M_\mathrm{T}}\xspace}
%===========================================
% Winkel 
%===========================================
\newcommand{\Amom}{\ensuremath{{\vartheta_\mathrm{e}}}\xspace}
\newcommand{\Ator}{\ensuremath{{\vartheta_\mathrm{T}}}\xspace}
%===========================================
% Reibbeiwerte 
%===========================================
\newcommand{\mg}{\ensuremath{{\mu_\mathrm{G}}}\xspace}
\newcommand{\mk}{\ensuremath{{\mu_\mathrm{K}}}\xspace}
\newcommand{\Fv}{\ensuremath{{F_\mathrm{M}}}\xspace}
\newcommand{\dtwo}{\ensuremath{{d_\mathrm{2}}}\xspace}
\newcommand{\dthree}{\ensuremath{{d_\mathrm{3}}}\xspace}
\newcommand{\dthread}{\ensuremath{{d}}\xspace}
\newcommand{\Dkm}{\ensuremath{{D_\mathrm{Km}}}\xspace}
\newcommand{\dw}{\ensuremath{{{d_\mathrm{W}}}}\xspace}
\newcommand{\dhe}{\ensuremath{{{d_\mathrm{ha}}}}\xspace}
\newcommand{\ds}{\ensuremath{{{d_\mathrm{S}}}}\xspace}
\newcommand{\Pg}{\ensuremath{{P}}\xspace}
\newcommand{\Gs}{\ensuremath{{G}}\xspace}
\newcommand{\len}{\ensuremath{l}\xspace}
\newcommand{\lenk}{\ensuremath{l_\mathrm{k}}\xspace}
\newcommand{\tm}{\ensuremath{{\mathrm{I}}}\xspace}
\newcommand{\fs}{\ensuremath{{{f_\mathrm{S}}}}\xspace}
\newcommand{\fp}{\ensuremath{{{f_\mathrm{P}}}}\xspace}
\newcommand{\fg}{\ensuremath{{{f_\mathrm{M}}}}\xspace}
\newcommand{\As}{\ensuremath{{{A_\mathrm{S}}}}\xspace}
\newcommand{\El}{\ensuremath{{{E}}}\xspace}
\newcommand{\Aflank}{\ensuremath{{{\alpha}}}\xspace}
\newcommand{\tors}{\ensuremath{{{\uptau}}}\xspace} 
\newcommand{\radi}{\ensuremath{{{r}}}\xspace}
\newcommand{\Wp}{\ensuremath{{{W_\mathrm{P}}}}\xspace}
\newcommand{\vx}{\ensuremath{{{V_\mathrm{x}}}}\xspace}
\newcommand{\vy}{\ensuremath{{{V_\mathrm{y}}}}\xspace}
\newcommand{\VX}{\ensuremath{{{\mathbf{V}_\mathrm{x}}}}\xspace}
\newcommand{\VY}{\ensuremath{{{\mathbf{V}_\mathrm{y}}}}\xspace}

%% 
% Wirkung der Momente 
\newcommand{\Mar}{\ensuremath{{\color{red!50!black}{\overset{\curvearrowright}{\mathrm{M}}_\mathrm{A}}}}\xspace}
\newcommand{\Mtr}{\ensuremath{{\color{red!50!black}{\overset{\curvearrowright}{\mathrm{M}}_\mathrm{T}}}}\xspace}
\newcommand{\Mgr}{\ensuremath{{\color{red!50!black}{\overset{\curvearrowright}{\mathrm{M}}_\mathrm{G}}}}\xspace}
\newcommand{\Mkr}{\ensuremath{{\color{red!50!black}{\overset{\curvearrowright}{\mathrm{M}}_\mathrm{K}}}}\xspace}
\newcommand{\Mtl}{\ensuremath{{\color{blue!50!black}{\overset{\curvearrowleft}{\mathrm{M}}_\mathrm{T}}}}\xspace}
\newcommand{\Mgl}{\ensuremath{{\color{blue!50!black}{\overset{\curvearrowleft}{\mathrm{M}}_\mathrm{G}}}}\xspace}
\newcommand{\Mkl}{\ensuremath{{\color{blue!50!black}{\overset{\curvearrowleft}{\mathrm{M}}_\mathrm{K}}}}\xspace}
\newcommand{\nxn}[1]{\num{#1}$\times$\num{#1}}
\newcommand{\myref}[2]{\href{#1}{#2}}
\DeclareSIUnit\rpm{rpm}
% Gewindemoment ? 
\def\mgtext{thread torque\xspace}
% Kopfmoment ? 
\def\mktext{head friction moment\xspace}
% Anziehdrehmoment 
\def\matext{tightening torque\xspace} 
\newcommand{\snull}{\ensuremath{\mathrm{S0}}\xspace}
\newcommand{\amag}{\ensuremath{\alpha_\mathrm{m}}\xspace}
\newcommand{\asx}{\ensuremath{\alpha_\mathrm{x}}\xspace}
\newcommand{\asy}{\ensuremath{\alpha_\mathrm{y}}\xspace}
\newcommand{\asum}{\ensuremath{\alpha_\mathrm{s}}\xspace}
\newcommand{\send}{\ensuremath{\mathrm{S15}}\xspace}
\newcommand{\rs}{\ensuremath{\mathbf{r}_\mathrm{s}}\xspace}
\newcommand{\rd}{\ensuremath{\mathbf{r}_\mathrm{d}}\xspace}
\newcommand{\rds}{\ensuremath{\mathbf{r}}\xspace}
\newcommand{\rabs}{\ensuremath{\mathrm{r}}\xspace}
\newcommand{\Hx}{\ensuremath{H_\mathrm{x}}\xspace}
\newcommand{\Hy}{\ensuremath{H_\mathrm{y}}\xspace}
\newcommand{\rx}{\ensuremath{r_\mathrm{x}}\xspace}
\newcommand{\ry}{\ensuremath{r_\mathrm{y}}\xspace}
\newcommand{\rz}{\ensuremath{r_\mathrm{z}}\xspace}
\newcommand{\mx}{\ensuremath{m_\mathrm{x}}\xspace}
\newcommand{\my}{\ensuremath{m_\mathrm{y}}\xspace}
\newcommand{\mz}{\ensuremath{m_\mathrm{z}}\xspace}
\newcommand{\ux}{\ensuremath{v_\mathrm{x}}\xspace}
\newcommand{\uy}{\ensuremath{v_\mathrm{y}}\xspace}
\newcommand{\Ux}{\ensuremath{V_\mathrm{x}}\xspace}
\newcommand{\Uy}{\ensuremath{V_\mathrm{y}}\xspace}
\newcommand{\Nend}{\ensuremath{\mathrm{N}}\xspace}
\newcommand{\Kend}{\ensuremath{\mathrm{K}}\xspace}
\newcommand{\nrun}{\ensuremath{n}\xspace}
\newcommand{\dphix}{\ensuremath{\Delta\varphi_\mathrm{x}}\xspace}
\newcommand{\dphiy}{\ensuremath{\Delta\varphi_\mathrm{y}}\xspace}
\newcommand{\dphixn}{\ensuremath{\Delta\varphi_\mathrm{xn}}\xspace}
\newcommand{\dphiyn}{\ensuremath{\Delta\varphi_\mathrm{yn}}\xspace}
\newcommand{\md}{\ensuremath{\mathbf{m}}\xspace}
\DeclareMathSymbol{\shortminus}{\mathbin}{AMSa}{"39}


\defbibenvironment{bibliography}
{\list
	{\printtext[labelnumberwidth]{%
			\printfield{labelprefix}%
			\printfield{labelnumber}}}
	{\setlength{\labelwidth}{\labelnumberwidth}%
		\setlength{\leftmargin}{\labelwidth}%
		\setlength{\labelsep}{1pt}%
		\addtolength{\leftmargin}{\labelsep}%
		\setlength{\itemsep}{\bibitemsep}%
		\setlength{\parsep}{\bibparsep}}%
	\renewcommand*{\makelabel}[1]{##1}}
{\endlist}
{\item}






