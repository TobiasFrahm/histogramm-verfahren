% !TeX encoding = utf8
%\documentclass[conference]{IEEEtran}
\documentclass[conference,10pt,a4paper]{IEEEtran}  
\IEEEoverridecommandlockouts
% The preceding line is only needed to identify funding in the first footnote. If that is unneeded, please comment it out.
%\pdfminorversion=7
\usepackage{stfloats} 
\usepackage{amsmath,amssymb,amsfonts}
\usepackage{algorithmic}
\usepackage{graphicx}
\usepackage{textcomp}
\usepackage{xcolor} 
\usepackage{cancel}


%\usepackage{helvet}
%\usepackage{tikz}
%\usepackage[scaled]{uarial}

\renewcommand*\familydefault{\sfdefault}

\usepackage[T1]{fontenc}
\usepackage[utf8]{inputenc}




\definecolor{color1}{RGB}{200,200,200}

\usepackage[hidelinks 	= 	true, 
draft 		=	false,	% all hypertext options are turned off
final 		=	true, 	% all hypertext options are turned on
raiselinks 	= 	true,	% Allow links to reflect real height - e.g. with pictures
breaklinks	=	false,	% false=Do not break a line within a link
backref		=	false,	% false=no backlinks in the bibliography
pagebackref =	false,	% false=no pagebacklinks in bibliography
linktocpage =	true,	% true=makes page-no linked and not text in TOC,LOF and LOT
%			colorlinks	=	true,	% true=Colors the text of links and anchors
colorlinks	=	false,	% true=Colors the text of links and anchors
%			linkcolor 	=	red, 	% Color for normal internal links.
%			anchorcolor =	black, 	% Color for anchor text.
%			citecolor 	=	green, 	% Color for bibliographical citations in text.
%			filecolor 	=	cyan, 	% Color for URLs which open local files.
%			menucolor 	=	red,	% Color for Acrobat menu items.
%			runcolor 	= filecolor,% Color for run links (launch annotations).
%			allcolors 	= 	red,	% Set all color options (without border and field options).
allcolors   =   color1, 
urlcolor 	=	color1,	% Color for linked URLs.
allbordercolors  =color1,	% Set all colors
%			citebordercolor =green,	% cite color
%			filebordercolor =cyan,	% file color
%			linkbordercolor =red,	% link color
%			urlbordercolor  =blue,	% url color
%			pdfborder 			= 0 0 0,% weder farbige links noch umrandung			
%			menubordercolor 	= red,	% set menu color
%			frenchlinks 		= false,% Use small caps instead of color for links.
bookmarks			= true,	% true=Bookmarks are added to the pdf
bookmarksopen 		= false,% false=Bookmarks not open when opening the pdf
bookmarksnumbered 	= true,	% true= Eintraege sind nummeriert 
%			pdfpagemode	=	FullScreen,	% File is opened in Full Screen
%			pdfstartview=	fit,	% Fit size when opening pdf
pdfpagelabels=	true,	% true = Roemische Zahlen usw werden dargestellt,
% false= fortlaufende nummerierung
]{hyperref}
 
 

 
 
% \title{Charakterisierung einer Schraubenverbindung mittels magnetischem Sensor-Array}
%\title{Characterization of a Bolted Joint with a Magnetic Sensor Array}
%\title{Magnetic Sensor Array for Determination of the Applied Preload of a Bolted Joint}
\title{\fontsize{16pt}{18pt}\selectfont\bfseries Histogramm-Verfahren für die Signalaussteuerung bei der Impedanzspektroskopie für Fahrzeugbatterien\\[12pt]
	\fontsize{10pt}{11.5pt}\selectfont\normalfont\itshape
	$\underline{\text{Tobias Frahm}}$ \textsuperscript{1}, 
		Florian Rittweger \textsuperscript{1}, Thorben Schüthe \textsuperscript{1}, Karl-Ragmar Riemschneider \textsuperscript{1} \\~\\
 \textsuperscript{1}Hochschule für Angewandte Wissenschaften Hamburg, Berliner Tor 7, 20099 Hamburg, GER \\
	
}
 

 

\usepackage{booktabs}
\usepackage{multirow} 
\usepackage{amsmath}
\usepackage{siunitx} 
\sisetup{locale = DE,  
	separate-uncertainty,  
	%	range-units = brackets,  
	range-units= repeat, 
	range-phrase= ~bis~ ,
	list-units = single,  
	%	per-mode=symbol,
	per-mode=symbol-or-fraction,
	round-precision=3}

\DeclareSIUnit\Nm{Nm}
\DeclareSIUnit\Nm{Nm}
\usepackage[left=2.5cm,right=2.5cm,top=2.5cm,bottom=2.5cm]{geometry}

\usepackage{mathtools}
\usepackage{xspace}
%\usepackage{tools-overview}


\usepackage[main=ngerman]{babel} 
\usepackage[babel]{csquotes} 
%\usepackage[]{caption} 
%\captionsetup{format=} 
\usepackage[figurename=Abb.,
			tablename=Tab.,
			format=hang,
			textfont=it,
			labelfont=it,
			labelsep=quad,
			skip=1em]{caption}  
%\setlength{\belowcaptionskip}{-1em}



%\usepackage[%
%style		= ieee,    
%defernumbers= true,      % to set different numeric types -e.g. A1...A2 B1...B2
%sortcase	= false,%		% false- keine unterscheidung zwischen gross und kleinschrift 
%bibencoding	= utf8,%
%doi 		= true, 
%isbn 		= false,
%subentry	= false,
%url 		= false,  
%backend		= biber,
%language    = {ngerman},
%maxcitenames=2,
%mincitenames=2,
%]{biblatex}				% um 8-bit zeichen zu erkennen, fuer umlaute 

%\DeclareFieldFormat{url}{Available\addcolon\space\url{#1}}
%\DefineBibliographyStrings{german}{
%	andothers = {{et\,al\adddot}},
%	and = \&,
%	page = {\:}
%} 
%\addto\extrasenglish{\languageshorthands{ngerman}} 
\usepackage{upgreek}
\usepackage{ltablex}