%%%%%%%%%%%%%%%%%%%%%%%%%%%%%%%%%%%%%%%%%%%%%%%%%%%%%%%%%%%%%%%%%%%%%%%
% Thema: Poster
% Autor: Musterman
% Datum: 12.05
%%%%%%%%%%%%%%%%%%%%%%%%%%%%%%%%%%%%%%%%%%%%%%%%%%%%%%%%%%%%%%%%%%%%%%%
%
\documentclass[a0,27pt,portrait]{a0poster}
%\documentclass[portrait,a0,posterdraft]{a0poster}
\pagestyle{empty}
\renewcommand{\figurename}{Abb.}
\renewcommand{\tablename}{Abb.}
%
%%%%%%%%%%%%%%%%%%%%%%%%%%%%%%%%%%%%%%%%%%%%%%%%%%%%%%%%%%%%%%%%%%%%%%%
% Paket zum Erzeugen der Spalten
\usepackage{multicol}
\usepackage{siunitx}
%%%%%%%%%%%%%%%%%%%%%%%%%%%%%%%%%%%%%%%%%%%%%%%%%%%%%%%%%%%%%%%%%%%%%%%
% Eingabekodierung
\usepackage[utf8]{inputenc}
\usepackage[T1]{fontenc}
\usepackage{ae} %wenn Ulaute nicht gehen
\usepackage{graphicx}
%\usepackage{subfigure}
\usepackage[ngerman]{babel}
\usepackage[hang]{caption}
\usepackage{floatrow}
\usepackage{multirow}
\addto\captionsngerman{\renewcommand{\figurename}{Fig.}}
%\addto\captionsngerman{\renewcommand{\subfigurename}{\sf }}
\addto\captionsngerman{\renewcommand{\tablename}{Tab.}}

%\begin{übler Trick um die Orinalüberschrift zu unterdrücken}
\addto{\centeringsngerman}{\renewcommand*{\refname}{}} %Literaturüberschfit unterdrücken
%\ende{übler Trick}
\usepackage[%
style=ieee,   
defernumbers=true,      % to set different numeric types -e.g. A1...A2 B1...B2
sortcase=false,%		% false- keine unterscheidung zwischen gross und kleinschrift 
bibencoding=utf8,%
doi = false, 
isbn = false,
subentry=true,
url = false,  
backend=biber,
maxcitenames=2,
mincitenames=1,
]{biblatex}				% um 8-bit zeichen zu erkennen, fuer umlaute
%%%%%%%%%%%%%%%%%%%%%%%%%%%%%%%%%%%%%%%%%%%%%%%%%%%%%%%%%%%%%%%%%%%%%%%
% für Farbe
\usepackage{color}
\definecolor{darkgreen}{rgb}{0,0.5,0}
\definecolor{darkblue}{rgb}{0,0,0.5}

%%%%%%%%%%%%%%%%%%%%%%%%%%%%%%%%%%%%%%%%%%%%%%%%%%%%%%%%%%%%%%%%%%%%%%%
% Mathepaket
\usepackage{amsmath}

%%%%%%%%%%%%%%%%%%%%%%%%%%%%%%%%%%%%%%%%%%%%%%%%%%%%%%%%%%%%%%%%%%%%%%%
% Ist für Boxen (wie das hier benutzte Ovalbox) zuständig
\usepackage{fancybox}

%%%%%%%%%%%%%%%%%%%%%%%%%%%%%%%%%%%%%%%%%%%%%%%%%%%%%%%%%%%%%%%%%%%%%%%
% Graphikpaket, ermöglicht png, jpg und pdf bilder
\usepackage{graphicx}

%%%%%%%%%%%%%%%%%%%%%%%%%%%%%%%%%%%%%%%%%%%%%%%%%%%%%%%%%%%%%%%%%%%%%%%
% Seiteneinstellungen
\renewcommand\baselinestretch{1.35}
\parskip=0.5\baselineskip

\parindent0mm %Einrücktiefe der ersten Zeile eines Absatzes
\topmargin-1cm
\marginparwidth0mm

%Ränder rechts/links
\oddsidemargin+10mm
\evensidemargin+10mm
\textwidth770mm
%\textheight1140mm

%%%%%%%%%%%%%%%%%%%%%%%%%%%%%%%%%%%%%%%%%%%%%%%%%%%%%%%%%%%%%%%%%%%%%%%
% Eigene Definitionen zur Erleichterung des Satzes
\newcommand{\spaltenbreite}{25}   % Spaltenbreite für Bilder
\newcommand{\bildbreite}{25cm}    % Einheitliche Bildbreite

%%%%%%%%%%%%%%%%%%%%%%%%%%%%%%%%%%%%%%%%%%%%%%%%%%%%%%%%%%%%%%%%%%%%%%%
% Box- und Spalteneinstellungen
\setlength{\fboxrule}{3.25mm} 	%Definiert die Linienstärke für nachfolgende fbox- und framebox-Befehle
\setlength{\fboxsep}{5mm}		%Abstand zwischen Rahmen und Text bei den /fbox und /framebox Befehlen.
\setlength{\columnsep}{15mm}	%Spaltenabstand
\setlength{\columnseprule}{0pt}	%Balken zwischen Spalten {0pt}->keine Balken

%%%%%%%%%%%%%%%%%%%%%%%%%%%%%%%%%%%%%%%%%%%%%%%%%%%%%%%%%%%%%%%%%%%%%%%
% Einheitslänge für picture-Umgebungen
%\setlength{\unitlength}{1.0cm}
\unitlength1cm

%%%%%%%%%%%%%%%%%%%%%%%%%%%%%%%%%%%%%%%%%%%%%%%%%%%%%%%%%%%%%%%%%%%%%%%
% Grafikpfad, hier liegen alle Bilder
%\graphicspath{{images/}}

%%%%%%%%%%%%%%%%%%%%%%%%%%%%%%%%%%%%%%%%%%%%%%%%%%%%%%%%%%%%%%%%%%%%%%%
% Zaehler fuer lineale. Sie werden gebraucht, wenn das Linealmacro included wird
\newcounter{skalax}
\newcounter{skalay}

%%%%%%%%%%%%%%%%%%%%%%%%%%%%%%%%%%%%%%%%%%%%%%%%%%%%%%%%%%%%%%%%%%%%%%%
% Workaround um figure-Umgebung in multicols zu nutzen
%
\makeatletter
\newenvironment{tablehere}
  {\def\@captype{table}}
  {}

\newenvironment{figurehere}
  {\def\@captype{figure}}
  {}
\makeatother


\newcommand{\fcap}[3]{\ffigbox{\caption{#1}}{\includegraphics[width=#2\textwidth]{#3}}}


\usepackage{upgreek}
\usepackage{amsmath}
\usepackage{siunitx} 
\sisetup{locale = US,  
	separate-uncertainty,  
	%	range-units = brackets,  
	range-units= repeat, 
	range-phrase= ~bis~ ,
	list-units = single,  
	%	per-mode=symbol,
	per-mode=symbol-or-fraction,
	round-precision=3}

\DeclareSIUnit\Nm{Nm}
\DeclareSIUnit\Nm{Nm}

\usepackage{mathtools}
\usepackage{xspace}
%\usepackage{tools-overview}
%===========================================
% Momente 
%===========================================
\newcommand{\MK}{\ensuremath{M_\mathrm{K}}\xspace}
\newcommand{\MA}{\ensuremath{M_\mathrm{A}}\xspace}
\newcommand{\MG}{\ensuremath{M_\mathrm{G}}\xspace}
\newcommand{\MT}{\ensuremath{M_\mathrm{T}}\xspace}
%===========================================
% Winkel 
%===========================================
\newcommand{\Amom}{\ensuremath{{\vartheta_\mathrm{e}}}\xspace}
\newcommand{\Ator}{\ensuremath{{\vartheta_\mathrm{T}}}\xspace}
%===========================================
% Reibbeiwerte 
%===========================================
\newcommand{\mg}{\ensuremath{{\mu_\mathrm{G}}}\xspace}
\newcommand{\mk}{\ensuremath{{\mu_\mathrm{K}}}\xspace}
\newcommand{\Fv}{\ensuremath{{F_\mathrm{M}}}\xspace}
\newcommand{\dtwo}{\ensuremath{{d_\mathrm{2}}}\xspace}
\newcommand{\dthree}{\ensuremath{{d_\mathrm{3}}}\xspace}
\newcommand{\dthread}{\ensuremath{{d}}\xspace}
\newcommand{\Dkm}{\ensuremath{{D_\mathrm{Km}}}\xspace}
\newcommand{\dw}{\ensuremath{{{d_\mathrm{W}}}}\xspace}
\newcommand{\dhe}{\ensuremath{{{d_\mathrm{ha}}}}\xspace}
\newcommand{\ds}{\ensuremath{{{d_\mathrm{S}}}}\xspace}
\newcommand{\Pg}{\ensuremath{{P}}\xspace}
\newcommand{\Gs}{\ensuremath{{G}}\xspace}
\newcommand{\len}{\ensuremath{l}\xspace}
\newcommand{\lenk}{\ensuremath{l_\mathrm{k}}\xspace}
\newcommand{\tm}{\ensuremath{{\mathrm{I}}}\xspace}
\newcommand{\fs}{\ensuremath{{{f_\mathrm{S}}}}\xspace}
\newcommand{\fp}{\ensuremath{{{f_\mathrm{P}}}}\xspace}
\newcommand{\fg}{\ensuremath{{{f_\mathrm{M}}}}\xspace}
\newcommand{\As}{\ensuremath{{{A_\mathrm{S}}}}\xspace}
\newcommand{\El}{\ensuremath{{{E}}}\xspace}
\newcommand{\Aflank}{\ensuremath{{{\alpha}}}\xspace}
\newcommand{\tors}{\ensuremath{{{\uptau}}}\xspace} 
\newcommand{\radi}{\ensuremath{{{r}}}\xspace}
\newcommand{\Wp}{\ensuremath{{{W_\mathrm{P}}}}\xspace}
\newcommand{\vx}{\ensuremath{{{V_\mathrm{x}}}}\xspace}
\newcommand{\vy}{\ensuremath{{{V_\mathrm{y}}}}\xspace}
\newcommand{\VX}{\ensuremath{{{\mathbf{V}_\mathrm{x}}}}\xspace}
\newcommand{\VY}{\ensuremath{{{\mathbf{V}_\mathrm{y}}}}\xspace}

%% 
% Wirkung der Momente 
\newcommand{\Mar}{\ensuremath{{\color{red!50!black}{\overset{\curvearrowright}{\mathrm{M}}_\mathrm{A}}}}\xspace}
\newcommand{\Mtr}{\ensuremath{{\color{red!50!black}{\overset{\curvearrowright}{\mathrm{M}}_\mathrm{T}}}}\xspace}
\newcommand{\Mgr}{\ensuremath{{\color{red!50!black}{\overset{\curvearrowright}{\mathrm{M}}_\mathrm{G}}}}\xspace}
\newcommand{\Mkr}{\ensuremath{{\color{red!50!black}{\overset{\curvearrowright}{\mathrm{M}}_\mathrm{K}}}}\xspace}
\newcommand{\Mtl}{\ensuremath{{\color{blue!50!black}{\overset{\curvearrowleft}{\mathrm{M}}_\mathrm{T}}}}\xspace}
\newcommand{\Mgl}{\ensuremath{{\color{blue!50!black}{\overset{\curvearrowleft}{\mathrm{M}}_\mathrm{G}}}}\xspace}
\newcommand{\Mkl}{\ensuremath{{\color{blue!50!black}{\overset{\curvearrowleft}{\mathrm{M}}_\mathrm{K}}}}\xspace}
